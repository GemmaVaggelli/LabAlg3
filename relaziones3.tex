\documentclass[12pt]{article}
\usepackage{graphicx}
\usepackage{geometry}
\usepackage{float}
\usepackage{amsmath}
\usepackage[italian]{babel}
\usepackage{listings,xcolor}
\definecolor{javared}{rgb}{0.6,0,0} % for strings
\definecolor{javagreen}{rgb}{0.25,0.5,0.35} % comments
\definecolor{javapurple}{rgb}{0.5,0,0.35} % keywords
\definecolor{javadocblue}{rgb}{0.25,0.35,0.75} % javadoc

\lstset{
  tabsize = 4, %% set tab space width
  showstringspaces = false, %% prevent space marking in strings, string is defined as the text that is generally printed directly to the console
  numbers = left, %% display line numbers on the left
  keywordstyle=\color{javagreen}\bfseries,
  stringstyle=\color{javared},
  commentstyle=\color{javagreen},
  morecomment=[s][\color{javadocblue}]{/*}*{*/},
  rulecolor = \color{black}, %% set frame color to avoid being affected by text color
  basicstyle = \small \ttfamily , %% set listing font and size
  breaklines = true, %% enable line breaking
  numberstyle = \tiny,
}



\begin{document}
\begin{titlepage}
   \begin{center}
       \vspace*{1cm}
	   \begin{Huge}
       \textbf{Valutazione sulle Componenti Strettamente Connesse nei grafi}
	   \end{Huge}
       \vspace{1.5cm}
       \\
       \begin{Large}
        Relazione esercizio 3
            
       \vspace{1.5cm}

       Laboratorio di Algoritmi e Strutture Dati
	   \vspace{3cm}
            
       Gemma Vaggelli \\
       6348717
            
       \vfill
     
    
            
       Ingegneria Informatica\\
       Università degli studi di Firenze
       \end{Large}
       
            
   \end{center}
\end{titlepage}

\section{Introduzione}

In questa relazione andrò a descrivere i risultati ottenuti dalla sperimentazione 
dell'algoritmo per la ricerca delle componenti fortemente connesse all'interno di un grafo orientato.\\
Vedremo cosa cambia all'aumentare del numero di nodi e della probabilità di 
presenza di archi tra i nodi.
Ci aspettiamo di vedere una convergenza a una sola componente connessa al crescere della probabilità e della dimensione. \


\section{Caratteristiche}

\subsection{Caratteristiche di un Grafo}

Un grafo orientato è una raccolta di due insiemi (V, E).\\
V è un insieme finito mentre E è una relazione binaria in V.\\
V è un insieme dei nodi e E l'insieme degli archi. Un arco è un insieme {u,v} dove u,v appartengono a V e u != v. Poiché è orientato ogni arco ha una direzione, quindi un nodo di partenza e un nodo di arrivo.\\

Un grafo orientato si dice \textbf{fortemente connesso} se due vertici qualsiasi sono raggiungibili l'uno dall'altro
\\
\begin{figure}[H]
\includegraphics[width = 10cm]{Graph}
\centering
\caption{Esempio di un grafo orientato fortemente connesso}
\end{figure}

\noindent
Un grafo viene rappresentato comunemente con:
\begin{itemize}
\item \textbf{Liste di adiacenza}
\item \textbf{Matrice di incidenza}
\end{itemize}
In questo esercizio ho scelto la prima rappresentazione.
\newpage


\subsection{Depht First Search}

Per trovare le componenti fortemente connesse è necessario introdurre un algoritmo di esplorazione in profondità del grafo. In particolare la DFS, che appena scopre un arco continua a esplorare da questo,
 e se necessario re-inizia da diversi nodi.
 
\begin{lstlisting}
DFS(G)
	for ogni vertice u in G.V
		u.color = WHITE
		u.pi = NIL
	time = 0
	for ogni vertice u in G.V
		if u.color = WHITE
			DFS - VISIT(G, u)
			
DFS - VISIT(G, u)
	time = time + 1
	u.d = time
	u.color = GRAY
	for ogni v in G.Adj[u]
		if v.color = WHITE
			v.pi = u
			DFS - VISIT(G, v)
	u.color = BLACK
	time = time + 1
	u.f = time
	
\end{lstlisting}

Questo algoritmo di esplorazione ha un costo in termini di tempo di $\Theta(|V| + |E|)$ \\\\

\subsection{Componenti Strettamente Connesse}

Dato un grafo G lo pseudoalgoritmo per trovare le componenti fortemente connesse di G è il seguente:\\

\noindent
Strongly Connected Components of G:
\begin{enumerate}
\item	DFS(G)
\item	si calcola la trasposta di G
\item	si esegue la DFS($G^T$) però prendendo i nodi in ordine decrescente rispetto ai tempi u.f
\item	si restituiscono le radici delle componenti fortemente connesse
\end{enumerate}
	
\noindent
Strongly Connected Components of G impiega $\Theta(|V| + |E|)$

\section{Documentazione}
Andiamo adesso a vedere come si comporta la SCC al crescere della probabilità di presenza di archi e 
al crescere della dimensione. In particolare andrò a differenziare la probabilità di un fattore $10$, quindi 
andrò a elencare un grafico per ogni incremento nella probabilità.
Gli esperimenti vengono svolti in questo caso 20 volte, e viene fatta una media per avere risultati più attendibili e precisi. I plot rappresentano i valori medi dei 20 esperimenti.
La dimensione dei grafi viene incrementata fino a un numero pari a 100 nodi.
Qui nelle figure vediamo sull'asse delle x la dimensione che aumenta di 1 nodo alla volta, mentre sulla y abbiamo due funzioni: la prima rappresenta il numero di SCC trovate per il grafo casuale con la propria probabilità di presenza di archi; la seconda è il tempo impiegato per trovare il numero di SCC.Nella colonna di destra andiamo a zoomare per vedere meglio i tempi.

\paragraph{Hardware}
I test sono stati eseguiti su un PC Desktop con sistema operativo Windows 10 Home a 64 bit, un processore Intel Core i5-7200U con 8Gb di RAM e Visual Studio Code come IDE.

\subsection{Grafici}

\begin{figure}[H]
\centering
    \makebox[\textwidth]{\makebox[1.25\textwidth]{%
    \begin{minipage}{.6\textwidth}
        \centering
        \includegraphics[width=1\textwidth]{prob0_scc_and_time}
        \caption{Numero di SCC per probabilità di presenza di archi dello 0\% }
        \label{fig:prob1}
    \end{minipage}\hfill
    \begin{minipage}{.6\textwidth}
        \centering
        \includegraphics[width=1\textwidth]{prob0_and_time}
        \caption{Numero di SCC per probabilità di presenza di archi dello 0\% }
    \end{minipage}}}
    \centering
\end{figure}

\begin{figure}[H]
\centering
    \makebox[\textwidth]{\makebox[1.25\textwidth]{%
    \begin{minipage}{.6\textwidth}
        \centering
        \includegraphics[width=1\textwidth]{prob10_scc_and_time}
        \caption{Numero di SCC per probabilità di presenza di archi del 10\% }
    \end{minipage}\hfill
    \begin{minipage}{.6\textwidth}
        \centering
        \includegraphics[width=1\textwidth]{prob10_and_time}
        \caption{Numero di SCC per probabilità di presenza di archi del 10\% }
    \end{minipage}}}
    \centering
\end{figure}
\newpage

\begin{figure}[H]
\centering
    \makebox[\textwidth]{\makebox[1.25\textwidth]
    \end{minipage}\hfill
    \begin{minipage}{.6\textwidth}
        \centering
        \includegraphics[width=1\textwidth]{prob20_and_time}
        \caption{Numero di SCC per probabilità di presenza di archi del 20\%}
    \end{minipage}}}
    \centering
\end{figure}

\begin{figure}[H]
\centering
    \makebox[\textwidth]{\makebox[1.25\textwidth]{%
    \begin{minipage}{.6\textwidth}
        \centering
        \includegraphics[width=1\textwidth]{prob30_scc_and_time}
        \caption{Numero di SCC per probabilità di presenza di archi del 30\% }
    \end{minipage}\hfill
    \begin{minipage}{.6\textwidth}
        \centering
        \includegraphics[width=1\textwidth]{prob30_and_time}
        \caption{Numero di SCC per probabilità di presenza di archi del 30\%}
    \end{minipage}}}
    \centering
\end{figure}

\begin{figure}[H]
\centering
    \makebox[\textwidth]{\makebox[1.25\textwidth]
    \end{minipage}\hfill
    \begin{minipage}{.6\textwidth}
        \centering
        \includegraphics[width=1\textwidth]{prob40_and_time}
        \caption{Numero di SCC per probabilità di presenza di archi del 40\% }
    \end{minipage}}}
    \centering
\end{figure}


\begin{figure}[H]
\centering
    \makebox[\textwidth]{\makebox[1.25\textwidth]{%
    \begin{minipage}{.6\textwidth}
        \centering
        \includegraphics[width=1\textwidth]{prob50_scc_and_time}
        \caption{Numero di SCC per probabilità di presenza di archi del 50\% }
    \end{minipage}\hfill
    \begin{minipage}{.6\textwidth}
        \centering
        \includegraphics[width=1\textwidth]{prob50_and_time}
        \caption{Numero di SCC per probabilità di presenza di archi del 50\% }
    \end{minipage}}}
    \centering
\end{figure}

\begin{figure}[H]
\centering
    \makebox[\textwidth]{\makebox[1.25\textwidth]{%
    \begin{minipage}{.6\textwidth}
        \centering
        \includegraphics[width=1\textwidth]{prob60_scc_and_time}
        \caption{Numero di SCC per probabilità di presenza di archi del 60\% }
    \end{minipage}\hfill
    \begin{minipage}{.6\textwidth}
        \centering
        \includegraphics[width=1\textwidth]{prob60_and_time}
        \caption{Numero di SCC per probabilità di presenza di archi del 60\%}
    \end{minipage}}}
    \centering
\end{figure}

\begin{figure}[H]
\centering
    \makebox[\textwidth]{\makebox[1.25\textwidth]
    \end{minipage}\hfill
    \begin{minipage}{.6\textwidth}
        \centering
        \includegraphics[width=1\textwidth]{prob70_and_time}
        \caption{Numero di SCC per probabilità di presenza di archi del 70\%}
    \end{minipage}}}
    \centering
\end{figure}


\begin{figure}[H]
\centering
    \makebox[\textwidth]{\makebox[1.25\textwidth]{%
    \begin{minipage}{.6\textwidth}
        \centering
        \includegraphics[width=1\textwidth]{prob80_scc_and_time}
        \caption{Numero di SCC per probabilità di presenza di archi dell' 80\% }
    \end{minipage}\hfill
    \begin{minipage}{.6\textwidth}
        \centering
        \includegraphics[width=1\textwidth]{prob80_and_time}
        \caption{Numero di SCC per probabilità di presenza di archi dell' 80\% }
    \end{minipage}}}
    \centering
\end{figure}

\begin{figure}[H]
\centering
    \makebox[\textwidth]{\makebox[1.25\textwidth]{%
    \begin{minipage}{.6\textwidth}
        \centering
        \includegraphics[width=1\textwidth]{prob90_scc_and_time}
        \caption{Numero di SCC per probabilità di presenza di archi del 90\% }
    \end{minipage}\hfill
    \begin{minipage}{.6\textwidth}
        \centering
        \includegraphics[width=1\textwidth]{prob90_and_time}
        \caption{Numero di SCC per probabilità di presenza di archi del 90\% }
    \end{minipage}}}
    \centering
\end{figure}

\begin{figure}[H]
\centering
    \makebox[\textwidth]{\makebox[1.25\textwidth]{%
    \begin{minipage}{.6\textwidth}
        \centering
        \includegraphics[width=1\textwidth]{prob100_scc_and_time}
        \caption{Numero di SCC per probabilità di presenza di archi del 100\% }
        \label{fig:prob2}
    \end{minipage}\hfill
    \begin{minipage}{.6\textwidth}
        \centering
        \includegraphics[width=1\textwidth]{prob100_and_time}
        \caption{Numero di SCC per probabilità di presenza di archi del 100\% }
    \end{minipage}}}
    \centering
\end{figure}


\begin{center}
\begin{figure}[H]
\includegraphics[width =\textwidth]{prob_and_scc}
\caption{Comportamento del grafo all' incremento della probabilità }
\label{fig:probscc}
\end{figure}
\end{center}

\section{Analisi e Conclusione}

Analizzando cosa succede al numero di Componenti Strettamente Connesse al crescere della probabilità di presenza di archi, il numero di componenti fortemente connesse dovrebbe tendere a 1. Questo dovrebbe avvenire poiché ad esempio a probabilità 100\% ogni nodo è connesso a tutti gli altri nodi. \\
Ciò si vede dai grafici precedenti (Figure \ref{fig:prob1} - \ref{fig:prob2} ) e da quello rappresentato in Figura \ref{fig:probscc} al crescere della probabilità.



Al tendere della probabilità a 0\%, invece, nessun nodo è collegato a un qualsiasi altro nodo, quindi 
il numero di SCC dovrebbe essere uguale al numero di nodi.\\
Anche questo fatto si nota dai grafici 


Guardando inoltre i tempi delle SCC, si nota che al crescere della probabilità
il tempo tende asintoticamente a $N^2$, vediamone il perché.
Il tempo di ogni SCC è un $\Theta(|V| + |E|)$.
Nel caso in cui la probabilità è del 100\% , e quindi si ha un numero di archi pari a 
$(N-1)*N$, il numero di archi è asintoticamente uguale a $N^2$.
Perciò al crescere della dimensione si ha un andamento asintotico del tempo impiegato pari a $N^2$



\end{document}